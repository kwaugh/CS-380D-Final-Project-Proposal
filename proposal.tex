\documentclass[10pt,twocolumn,letterpaper]{article}

\usepackage{cvpr}
\usepackage{times}
\usepackage{epsfig}
\usepackage{graphicx}
\usepackage{amsmath}
\usepackage{amssymb}

% Include other packages here, before hyperref.

% If you comment hyperref and then uncomment it, you should delete
% egpaper.aux before re-running latex.  (Or just hit 'q' on the first latex
% run, let it finish, and you should be clear).
\usepackage[breaklinks=true,bookmarks=false]{hyperref}

\cvprfinalcopy

\def\cvprPaperID{****} % *** Enter the CVPR Paper ID here
\def\httilde{\mbox{\tt\raisebox{-.5ex}{\symbol{126}}}}

\setcounter{page}{1}
\begin{document}

%%%%%%%%% TITLE
\title{CS 380D Distributed Systems Final Project Proposal}

\author{Keivaun Waugh\\
University of Texas at Austin\\
{\tt\small keivaunwaugh@utexas.com}
\and
Paul Choi\\
University of Texas at Austin\\
{\tt\small choipaul96@gmail.com}
\and
Jo Bridgwater\\
University of Texas at Austin\\
{\tt\small jobridgwater@utexas.edu}
}

\maketitle

\begin{abstract}
    In this project, we aim to evaluate the performance of CockroachDB on top of
    PebblesDB. The open source version of CockroachDB currently available is
    built on top of RocksDB. PebblesDB is a key-value store developed at UT
    that has some performance trade-offs with RocksDB, with its main
    contribution being its faster write performance. In addition to measuring
    the performance change, we will investigate what changes, if any, need to
    be made to PebblesDB or its configuration to increase the performance of
    CockroachDB.
\end{abstract}

\section{Introduction}
CockroachDB is a distributed SQL database with strong consistency guarantees.
CockroachDB offers many enhancements on top of a traditional key-value store
including replication and recovery as well as other standard features that are
provided by a database frontend. By replacing RocksDB, the current key-value
store that CockroachDB uses as a storage backend, with PebblesDB, we aim to see
if the performance increases that PebblesDB brings can also apply to
CockroachDB in a distributed setting.

RocksDB is a key-value store based on LevelDB built using a log-structured
merge tree, which exploits the relatively higher speed of sequential disk I/O
by making writes to disk append-only rather than random. This is achieved by
batching writes within an in-memory store and periodically flushing these
writes to files stored on disk called sstables. Sstables are organized in a
pyramid-like structure called levels. Keys are sorted within each sstable, and
non-overlapping across all sstables in each level. Reads search for the target
key starting from the first level and moving downward. This system has low
write latency at the cost of higher read latency, since reads may have to
search through multiple levels to reach the target key. This downside is
alleviated somewhat by periodically compacting the sstables within a level into
larger files in the next level.

PebblesDB is also a key-value store based on LevelDB, but it decreases
write-amplification and increases write throughput by using a fragmented
log-structured merge tree (FLSM). In FLSMs, groups of sstables in each level
are allowed to have overlapping key spaces within each group. When managed in
an intelligent manner, this can eliminate some of the redundancy in writing
keys from one level to the next during compaction in traditional LSTMs.
However, it adds some read latency since a read may have to read multiple
sstables within a group in a given level to find the key, as key ranges of the
sstables may overlap.

%------------------------------------------------------------------------------

\section{Technical Plan}
Our contributions will be three-fold. Because CockroachDB does not have a clear
separation of logic between its use of RocksDB as a storage backend versus any
abstract key-value store, we will evaluate the practicality of refactoring the
CockroachDB source to support this. This will make it easier in the future to
evaluate other key-value stores as storage backends for CockroachDB.

Additionally, because PebblesDB and RocksDB do not share the same API, we will
write a shim layer to perform API translation. RocksDB was built on work from
LevelDB, so it is likely that they share a comparable API, but we have not yet
evaluated their API compatibility.  This approach appears to be more
straightforward than modifying CockroachDB directly to use the new API.

Finally, we will try to optimize PebblesDB or explore the impact of alternate
configurations for PebblesDB to utilize its performance gains over RocksDB to
find similar performance gains for CockroachDB.

%------------------------------------------------------------------------------

\section{Evaluation Plan}
In order to evaluate CockroachDB's performance when using PebblesDB versus
RocksDB, we will use the widely used Yahoo! Cloud Serving Benchmark (YCSB). We
will also look for other benchmarks that are commonly used to evaluate database
performance and run them on both systems if we find that they address an area
which YCSB does not. We will investigate any differences in performance on
write-, update-, read-, and delete-heavy workloads.

In addition to evaluating performance after successfully integrating PebblesDB
as the storage backend, we will explore changes which can be made to PebblesDB
or CockroachDB which yield further improvement, and quantify the difference in
performance.

\section{Materials}
In order to perform a thorough evaluation, we will need access to multiple
machines with SSDs to test the distributed performance of CockroachDB. We
require SSDs rather than hard drives in order to properly compare performance
between RocksDB and PebblesDB, as RocksDB is optimized for flash storage. We
would like to explore the use of either existing test machines in the UT
systems lab or machines on a cloud hosting provider possibly using credits
provided to the class by Google, Amazon, etc.

%------------------------------------------------------------------------------

\end{document}
